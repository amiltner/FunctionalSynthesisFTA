\documentclass[acmsmall,nonacm]{acmart}

\AtBeginDocument{%
  \providecommand\BibTeX{{%
    \normalfont B\kern-0.5em{\scshape i\kern-0.25em b}\kern-0.8em\TeX}}}

\usepackage{pifont}% http://ctan.org/pkg/pifont
\newcommand{\cmark}{\ding{51}}%
\newcommand{\xmark}{\ding{55}}%

\newcommand{\highlightBlue}[1]{\textcolor{blue}{\textbf{#1}}}
\newcommand{\highlightRed}[1]{\textcolor{red}{\textbf{#1}}}

\newcommand{\snsMyth}
  {\ensuremath{\textsc{Smyth}}}
\newcommand{\myth}
  {\ensuremath{\textsc{Myth}}}
\newcommand{\leon}
  {\ensuremath{\textsc{Leon}}}
\newcommand{\synquid}
  {\ensuremath{\textsc{Synquid}}}

\newcommand{\experimentTableSize}
  {\scriptsize}
\newcommand{\experimentCaptionSize}
  {\footnotesize}

\newcommand{\vsepBeforeCaption}{\vspace{0.05in}}

\begin{document}

\title{Smyth Experiment Tables}
%% \subtitle{May 2020}
\maketitle

\vspace{0.30in}
\noindent
\textbf{Figure 10 Results.}
%
The following page replicates Figure 10 from our paper.
%
In Tables 1, 2, and 3:
%
the \textit{Ours} column shows the results obtained and summarized by
the scripts on our machine;
%
the \textit{Fig. 10} columns displays these results, using labels and
colors to explain not-applicable or failed tasks; and
%
the \textit{Yours} column shows the results that you obtained
on your machine; differences from ours are highlighted in red.
%
If you have not run the experiments, the column should be filled with dots.


\vspace{0.30in}
%
\noindent
%
\textbf{\leon{} and \synquid{} Benchmarks (\texttt{experiments/exp-4-logic/}):}
%
For Experiment 4, we wrote a script
%
(\texttt{generate-benchmarks.py})
%
to generate \leon{} and \synquid{} tasks
%
(\texttt{generated/}),
%
which we copied and pasted into the respective web editors.
%
Our labeled results are in \texttt{results/}.
%
The helper script \texttt{show-results.sh} summarizes the labels,
and also includes comments with some shell commands we found useful during
this experiment.
%
The sketching benchmarks we tried
can be found in \texttt{sketches/}.

\vspace{0.30in}
\noindent
\textbf{Figure 20 Results.}
%
The final page replicates Figure 20 from the appendix to our paper.
%
You can run Experiments 5 and 6 on your machine, but our scripts do not
generate tables showing differences as for Figure 10.

\clearpage

\setcounter{figure}{9}
\newcommand{\numBenchmarks}{38}
\newcommand{\numBenchmarksAll}{43}

\newcommand{\pctFewerExamplesTopOne}{61\%}
\newcommand{\pctFewerExamplesTopOneUpperBound}{66\%}

\newcommand{\numBenchmarksRecursiveAll}{32}
\newcommand{\numBenchmarksRecursive}{27}
\newcommand{\numBenchmarksBase}{25}

\newcommand{\pctFewerExamplesBaseCaseNoSketch}{57\%}
\newcommand{\pctFewerExamplesBaseCaseStrategy}{46\%}

\newcommand{\maxExamplesBase}{7}
\newcommand{\avgExamplesBase}{2.12}

\newcommand{\benchmarkName}[1]
  {#1}
\newcommand{\benchmarkNameBool}[1]
  {\benchmarkName{bool\_#1}}
\newcommand{\benchmarkNameList}[1]
  {\benchmarkName{list\_#1}}
\newcommand{\benchmarkNameNat}[1]
  {\benchmarkName{nat\_#1}}
\newcommand{\benchmarkNameTree}[1]
  {\benchmarkName{tree\_#1}}

%% \newcommand{\benchmarkTypeGap}
%%   {&&&&&&&&&&}

\definecolor{skippedColor}{HTML}{c0c0c0}

\newcommand{\labelColorSkipped}[1]
  {\textcolor{skippedColor}{#1}}
\newcommand{\labelColorFailed}[1]
  {\textcolor{orange}{#1}}
\newcommand{\labelTimeout}
  {\labelColorFailed{timeout}}
\newcommand{\labelOverspec}
  {\labelColorFailed{overspec}}
\newcommand{\labelRandomFailed}
  %% {\labelColorFailed{failed}}
  {\labelColorFailed{(---,---)}}
\newcommand{\labelIncorrect}
  {\labelColorFailed{incorrect}}
\newcommand{\labelBlank}
  %% {---}
  {$\bullet$}
\newcommand{\labelBlankOneFailed}
  {\labelColorSkipped{\labelBlank$^1$}}
\newcommand{\labelBlankHigherOrder}
  {\labelColorSkipped{\labelBlank$^2$}}
\newcommand{\labelBlankNonRec}
  {\labelColorSkipped{\labelBlank$^3$}}
\newcommand{\labelBlankSameExpertExamples}
  {\labelColorSkipped{\labelBlank$^4$}}
\newcommand{\labelRandomTime}[1]
  %% {\textcolor{orange}{#1}}
  %% {#1}
  {t=#1}

\newcommand{\benchmarkExperimentOne}[3]
  {&{#1}&{#3}}
\newcommand{\benchmarkExperimentOneFailedTimeOut}[1]
  {&{#1}&{\scriptsize{timeout}}}
\newcommand{\benchmarkExperimentOneFailedOverSpecialized}[1]
  {&{#1}&{\scriptsize{overspec}}}
\newcommand{\benchmarkExperimentOneFailedNoSolutions}[1]
  {&{#1}&{\scriptsize{none}}}
\newcommand{\benchmarkExperimentThreeFailedTimeOut}
  {&{\scriptsize{timeout}}}
\newcommand{\benchmarkExperimentThreeFailedOverSpecialized}
  {&{\scriptsize{overspec}}}
\newcommand{\benchmarkExperimentTwo}[9]
  {&{#1}
  }
%% TODO rename: used for experiment three also
\newcommand{\benchmarkExperimentTwoRand}[3]
  {&(#1, #2)$^{#3}$}
\newcommand{\benchmarkExperimentTwoRandFailedHigherOrder}
  {&\blankEntry}
\newcommand{\benchmarkExperimentTwoRandFailedTimeout}
  {&\scriptsize{timeout}}
%% TODO rename
\newcommand{\benchmarkExperimentTwoRandFailedNoNinety}
  {&\scriptsize{failed}}
\newcommand{\benchmarkExperimentThree}[9]
  {&{#1}}

\newcommand{\insideOut}
  %% {*}
  {}
\newcommand{\upperBound}
  {*}
\newcommand{\blankEntry}
  {---}
\newcommand{\benchmarkExperimentTwoBlank}
  {&\blankEntry}
\newcommand{\benchmarkExperimentThreeBlank}
  {&\blankEntry}
\newcommand{\displayPct}[1]
  {\phantom{0 (}{#1}\phantom{)}}
\newcommand{\displayPctUpperBound}[1]
  {\phantom{0 (}{#1}\upperBound}

\newcommand{\leonquidBlank}
  {\textcolor{gray}{---}}
\newcommand{\leonquidCorrectNoPhantom}
  {\textcolor{gray}{\cmark}}
\newcommand{\leonquidCorrect}
  {\textcolor{gray}{\cmark\phantom{$^1$}}}
\newcommand{\leonquidIncorrect}
  {\textcolor{orange}{\xmark$^1$}}
\newcommand{\leonquidError}
  {\textcolor{orange}{\xmark$^0$}}
\newcommand{\leonquidHigherOrderFunc}
  {\textcolor{orange}{\xmark$^2$}}
\newcommand{\synquidDatatypeAxiomsNoPhantom}
  {\textcolor{orange}{\textbf{?}}}
\newcommand{\synquidDatatypeAxioms}
  {\textcolor{orange}{\textbf{?\ }}}

\begin{figure}

\experimentTableSize

\begin{tabular}{l|cc|cc|cc||cc|cc}
& \multicolumn{6}{c||}{\textbf{\snsMyth{}}}
& \multicolumn{2}{c}{\textbf{\leon{}}}
& \multicolumn{2}{|c}{\textbf{\synquid{}}}
\\\hline
\multicolumn{1}{r|}{\textbf{Experiment}} &
\multicolumn{2}{c|}{\textbf{1}} &
\textbf{2a} & \textbf{2b} & \textbf{3a} & \textbf{3b}
& \multicolumn{2}{c|}{\textbf{4}} & \multicolumn{2}{c}{\textbf{4}}
\\\hline
\multicolumn{1}{r|}{{Sketch / Objective}} &
\multicolumn{2}{c|}{\textit{None / Top-1}} &
\multicolumn{2}{c|}{\textit{None / Top-1}} &
\multicolumn{2}{c||}{\textit{Base Case / Top-1-R}}
%% \multicolumn{1}{r|}{{Sketch}} &
%% \multicolumn{4}{c|}{\textit{None}} &
%% \multicolumn{2}{c||}{\textit{Base Case}}
& \multicolumn{2}{c|}{}
\\\hline
%% \multicolumn{1}{r|}{{\#Benchmarks}} &
%% \multicolumn{4}{c|}{\textit{\numBenchmarks{}/\numBenchmarksAll{} \myth{} benchmarks}} &
%% \multicolumn{2}{c||}{\textit{\numBenchmarksBase{}/\numBenchmarksRecursive{} rec. benchmarks}}
%% \multicolumn{4}{c|}{\textit{\numBenchmarksAll{} \myth{} benchmarks}} &
%% \multicolumn{2}{c||}{\textit{\numBenchmarksRecursiveAll{} recursive benchmarks}}
%% & \multicolumn{2}{c|}{}
%% \\\hline
%% \multicolumn{1}{r|}{{Objective}} &
%% \multicolumn{2}{c|}{\makebox[0.38in]{\textit{Top-1}}} &
%% \multicolumn{2}{c|}{\makebox[0.38in]{\textit{Top-1}}} &
%% \multicolumn{2}{c||} {\makebox[0.38in]{\textit{Top-1-R}}}
%% & \multicolumn{2}{c|}{}
%% \\\hline
\textbf{Name} &
\textbf{Expert} & \textbf{Time} &
\textbf{Expert} & \textbf{Random} &
\textbf{Expert} & \textbf{Random} &
\textbf{1} & \textbf{2a} &
\textbf{1} & \textbf{2a}
\\
&
& &
& {(50\%, 90\%)} &
& {(50\%, 90\%)} &
& &
\\
&&&&&&&&&&\\
bool\_band&4&0.004&3 (75\%)&(4,4)$^{}$&\labelBlankNonRec&\labelBlankNonRec&\leonquidCorrect&\leonquidCorrect&\leonquidCorrect&\leonquidCorrect\\
bool\_bor&4&0.003&3 (75\%)&(4,4)$^{}$&\labelBlankNonRec&\labelBlankNonRec&\leonquidCorrect&\leonquidCorrect&\leonquidCorrect&\leonquidCorrect\\
bool\_impl&4&0.004&3 (75\%)&(4,4)$^{}$&\labelBlankNonRec&\labelBlankNonRec&\leonquidCorrect&\leonquidCorrect&\leonquidCorrect&\leonquidCorrect\\
bool\_neg&2&0.001&2 (100\%)&(2,2)$^{}$&\labelBlankNonRec&\labelBlankNonRec&\leonquidCorrect&\labelBlankSameExpertExamples&\leonquidCorrect&\labelBlankSameExpertExamples\\
bool\_xor&4&0.009&4 (100\%)&(4,4)$^{}$&\labelBlankNonRec&\labelBlankNonRec&\leonquidCorrect&\labelBlankSameExpertExamples&\leonquidCorrect&\labelBlankSameExpertExamples\\
&&&&&&&&&&\\
list\_append&6&0.008&4 (67\%)&(3,4)$^{}$&1+1 (33\%)&(1+3,1+4)$^{}$&\leonquidCorrect&\leonquidIncorrect&\leonquidCorrect&\leonquidIncorrect\\
list\_compress&13&\labelTimeout&\labelBlankOneFailed&\labelBlankOneFailed&\labelBlankOneFailed&\labelBlankOneFailed&\labelBlankOneFailed&\labelBlankOneFailed&\labelBlankOneFailed&\labelBlankOneFailed\\
list\_concat&6&0.010&3 (50\%)&(2,4)$^{}$&\labelIncorrect&(1+3,1+5)$^{}$&\leonquidCorrect&\leonquidIncorrect&\leonquidIncorrect&\leonquidIncorrect\\
list\_drop&11&0.092&5 (45\%)&(6,9)$^{}$&1+2 (27\%)&\labelColorFailed{(1+7,$\downarrow$)$^{}$}&\leonquidCorrect&\leonquidCorrect&\leonquidCorrect&\leonquidError\\
list\_even\_parity&7&\labelOverspec&\labelBlankOneFailed&\labelRandomFailed&\labelBlankOneFailed&\labelRandomFailed&\labelBlankOneFailed&\labelBlankOneFailed&\labelBlankOneFailed&\labelBlankOneFailed\\
list\_filter&\phantom{*}9*&0.144&5 (56\%)&\labelBlankHigherOrder&1+4 (56\%)&\labelBlankHigherOrder&\leonquidHigherOrderFunc&\leonquidHigherOrderFunc&\leonquidHigherOrderFunc&\leonquidHigherOrderFunc\\
list\_fold&9&0.838&3 (33\%)&\labelBlankHigherOrder&1+3 (44\%)&\labelBlankHigherOrder&\leonquidHigherOrderFunc&\leonquidHigherOrderFunc&\leonquidHigherOrderFunc&\leonquidHigherOrderFunc\\
list\_hd&3&0.003&2 (67\%)&(2,3)$^{}$&\labelBlankNonRec&\labelBlankNonRec&\leonquidCorrect&\leonquidCorrect&\leonquidCorrect&\leonquidCorrect\\
list\_inc&4&0.018&2 (50\%)&(2,2)$^{}$&\labelBlankNonRec&\labelBlankNonRec&\leonquidCorrect&\leonquidCorrect&\leonquidError&\leonquidIncorrect\\
list\_last&6&0.007&4 (67\%)&(5,9)$^{}$&1+2 (50\%)&(1+5,1+10)$^{}$&\leonquidCorrect&\leonquidCorrect&\leonquidCorrect&\leonquidError\\
list\_length&3&0.002&3 (100\%)&(3,4)$^{}$&1+1 (67\%)&(1+2,1+2)$^{}$&\leonquidCorrect&\labelBlankSameExpertExamples&\leonquidCorrect&\labelBlankSameExpertExamples\\
list\_map&8&0.049&4 (50\%)&\labelBlankHigherOrder&1+2 (38\%)&\labelBlankHigherOrder&\leonquidHigherOrderFunc&\leonquidHigherOrderFunc&\leonquidHigherOrderFunc&\leonquidHigherOrderFunc\\
list\_nth&13&0.124&5 (38\%)&(7,14)$^{}$&1+2 (23\%)&(1+7,1+15)$^{}$&\leonquidCorrect&\leonquidCorrect&\leonquidCorrect&\leonquidError\\
list\_pairwise\_swap&7&0.634&5 (71\%)&\labelTimeout&\labelOverspec&\labelTimeout&\leonquidCorrect&\leonquidCorrect&\leonquidError&\leonquidError\\
list\_rev\_append&5&0.107&3 (60\%)&(5,8)$^{}$&1+2 (60\%)&(1+3,1+4)$^{}$&\leonquidCorrect&\leonquidCorrect&\leonquidError&\leonquidError\\
list\_rev\_fold&5&0.035&2 (40\%)&(2,4)$^{}$&\labelBlankNonRec&\labelBlankNonRec&\leonquidCorrect&\leonquidCorrect&\leonquidError&\leonquidError\\
list\_rev\_snoc&5&0.010&3 (60\%)&(3,6)$^{}$&1+1 (40\%)&(1+2,1+4)$^{}$&\leonquidCorrect&\leonquidCorrect&\leonquidIncorrect&\leonquidError\\
list\_rev\_tailcall&8&0.008&3 (38\%)&(3,4)$^{}$&1+1 (25\%)&(1+3,1+5)$^{}$&\leonquidIncorrect&\leonquidCorrect&\leonquidCorrect&\leonquidIncorrect\\
list\_snoc&8&0.012&3 (38\%)&(3,4)$^{}$&1+1 (25\%)&(1+3,1+4)$^{}$&\leonquidCorrect&\leonquidCorrect&\leonquidCorrect&\leonquidError\\
list\_sort\_sorted\_insert&7&0.015&3 (43\%)&(3,6)$^{}$&1+1 (29\%)&(1+2,1+4)$^{}$&\leonquidCorrect&\leonquidCorrect&\leonquidError&\leonquidIncorrect\\
list\_sorted\_insert&12&2.902&7 (58\%)&\labelTimeout&1+7 (67\%)&\labelTimeout&\leonquidError&\leonquidError&\leonquidError&\leonquidError\\
list\_stutter&3&0.003&2 (67\%)&(3,3)$^{}$&1+1 (67\%)&(1+2,1+3)$^{}$&\leonquidCorrect&\leonquidCorrect&\leonquidCorrect&\leonquidIncorrect\\
list\_sum&3&0.029&2 (67\%)&(2,2)$^{}$&\labelBlankNonRec&\labelBlankNonRec&\leonquidCorrect&\leonquidIncorrect&\leonquidError&\leonquidError\\
list\_take&12&0.065&5 (42\%)&(6,9)$^{}$&1+3 (33\%)&(1+7,1+16)$^{}$&\leonquidCorrect&\leonquidCorrect&\leonquidCorrect&\leonquidError\\
list\_tl&3&0.002&2 (67\%)&(2,3)$^{}$&\labelBlankNonRec&\labelBlankNonRec&\leonquidCorrect&\leonquidCorrect&\leonquidCorrect&\leonquidCorrect\\
&&&&&&&&&&\\
nat\_add&9&0.006&4 (44\%)&(5,6)$^{}$&1+1 (22\%)&(1+3,1+4)$^{}$&\leonquidCorrect&\leonquidCorrect&\leonquidCorrect&\leonquidIncorrect\\
nat\_iseven&4&0.003&3 (75\%)&(4,4)$^{}$&1+2 (75\%)&(1+3,1+4)$^{}$&\leonquidCorrect&\leonquidCorrect&\leonquidCorrect&\leonquidError\\
nat\_max&9&0.041&9 (100\%)&(8,12)$^{}$&1+4 (56\%)&(1+8,1+12)$^{}$&\leonquidIncorrect&\labelBlankSameExpertExamples&\leonquidCorrect&\labelBlankSameExpertExamples\\
nat\_pred&3&0.001&2 (67\%)&(2,3)$^{}$&\labelBlankNonRec&\labelBlankNonRec&\leonquidCorrect&\leonquidCorrect&\leonquidCorrect&\leonquidCorrect\\
&&&&&&&&&&\\
tree\_binsert&20&\labelTimeout&\labelBlankOneFailed&\labelBlankOneFailed&\labelBlankOneFailed&\labelBlankOneFailed&\labelBlankOneFailed&\labelBlankOneFailed&\labelBlankOneFailed&\labelBlankOneFailed\\
tree\_collect\_leaves&6&0.074&3 (50\%)&(3,4)$^{}$$^{\labelRandomTime{3}}$&1+2 (50\%)&(1+3,1+3)$^{}$&\leonquidCorrect&\leonquidCorrect&\leonquidIncorrect&\leonquidIncorrect\\
tree\_count\_leaves&7&2.660&3 (43\%)&\labelTimeout&1+1 (29\%)&\labelTimeout&\leonquidCorrect&\leonquidCorrect&\leonquidError&\leonquidError\\
tree\_count\_nodes&6&0.351&3 (50\%)&\labelColorFailed{(4,$\downarrow$)$^{}$}$^{\labelRandomTime{10}}$&1+2 (50\%)&(1+3,1+5)$^{}$$^{\labelRandomTime{3}}$&\leonquidCorrect&\leonquidCorrect&\leonquidIncorrect&\leonquidError\\
tree\_inorder&5&0.123&4 (80\%)&(3,4)$^{}$&1+2 (60\%)&(1+3,1+4)$^{}$&\leonquidCorrect&\leonquidCorrect&\leonquidIncorrect&\leonquidError\\
tree\_map&7&0.061&4 (57\%)&\labelBlankHigherOrder&1+3 (57\%)&\labelBlankHigherOrder&\leonquidHigherOrderFunc&\leonquidHigherOrderFunc&\leonquidHigherOrderFunc&\leonquidHigherOrderFunc\\
tree\_nodes\_at\_level&11&\labelTimeout&\labelBlankOneFailed&\labelBlankOneFailed&\labelBlankOneFailed&\labelBlankOneFailed&\labelBlankOneFailed&\labelBlankOneFailed&\labelBlankOneFailed&\labelBlankOneFailed\\
tree\_postorder&20&\labelTimeout&\labelBlankOneFailed&\labelBlankOneFailed&\labelBlankOneFailed&\labelBlankOneFailed&\labelBlankOneFailed&\labelBlankOneFailed&\labelBlankOneFailed&\labelBlankOneFailed\\
tree\_preorder&5&0.153&3 (60\%)&(3,4)$^{}$$^{\labelRandomTime{3}}$&1+2 (60\%)&(1+3,1+3)$^{}$&\leonquidCorrect&\leonquidCorrect&\leonquidIncorrect&\leonquidIncorrect\\
&&&&&&&&&&\\

\hline
\textbf{Averages} &
&
&
\displayPctUpperBound{\pctFewerExamplesTopOne} &
&
\displayPct{\phantom{1+}\pctFewerExamplesBaseCaseStrategy} &
&&
\end{tabular}

\vsepBeforeCaption
  \captionsetup{justification=centering}
  \caption{
    Experiments.
    \\
      \textbf{Top-1(-R)}:
      1st (recursive) solution valid.
    \textbf{Time}:
       Average of 10 runs, in seconds.
    \\
    \textbf{2a Average}:
      \pctFewerExamplesTopOne{} for \numBenchmarks{} non-blank rows.
      (*Upper bound: \pctFewerExamplesTopOneUpperBound{} for all
     \numBenchmarksAll{} rows.)
    \\
    \textbf{3a Average}:
      \pctFewerExamplesBaseCaseStrategy{} for
      \numBenchmarksBase{} non-blank, non-error rows.
  }
\label{fig:experiments}
\end{figure}

\setcounter{figure}{0}

\begin{table}

\experimentTableSize

\begin{tabular}{l|cccccc}
& \multicolumn{6}{c}{\textbf{Experiment 1}} \\\hline
\textbf{Name} &
\textbf{Expert} & \textbf{Expert} & \textbf{Expert} &
\textbf{Time} & \textbf{Time} & \textbf{Time} \\
&
\textit{Fig. 10} & \textit{Ours} & \textit{Yours} &
\textit{Fig. 10} & \textit{Ours} & \textit{Yours} \\
\input{generated/table-1-data}
\end{tabular}

\vspace{0.10in}

\caption{Experiment 1.}

\end{table}

\begin{table}

\experimentTableSize

\begin{tabular}{l|cccccc}
& \multicolumn{3}{c}{\textbf{Experiment 2a}}
& \multicolumn{3}{c}{\textbf{Experiment 2b}} \\\hline
\textbf{Name} &
\textbf{Expert} & \textbf{Expert} & \textbf{Expert} &
\textbf{Random} & \textbf{Random} & \textbf{Random} \\
&
\textit{Fig. 10} & \textit{Ours} & \textit{Yours} &
\textit{Fig. 10} & \textit{Ours} & \textit{Yours} \\
\input{generated/table-2-data}
\end{tabular}

\vspace{0.10in}

\caption{Experiment 2.}

\end{table}

\begin{table}

\experimentTableSize

\begin{tabular}{l|cccccc}
& \multicolumn{3}{c}{\textbf{Experiment 3a}}
& \multicolumn{3}{c}{\textbf{Experiment 3b}} \\\hline
\textbf{Name} &
\textbf{Expert} & \textbf{Expert} & \textbf{Expert} &
\textbf{Random} & \textbf{Random} & \textbf{Random} \\
&
\textit{Fig. 10} & \textit{Ours} & \textit{Yours} &
\textit{Fig. 10} & \textit{Ours} & \textit{Yours} \\
\input{generated/table-3-data}
\end{tabular}

\vspace{0.10in}

\caption{Experiment 3.}

\end{table}


\setcounter{figure}{19}
\begin{figure}[h]

\experimentTableSize

\begin{tabular}{l|cc|cc}
& \multicolumn{4}{c}{\textbf{\snsMyth{}}}
\\\hline
\multicolumn{1}{r|}{\textbf{Experiment}} &
\textbf{5a} & \textbf{5b} & \textbf{6a} & \textbf{6b}
\\\hline
\multicolumn{1}{r|}{{Sketch / Objective}} &
\multicolumn{2}{c|}{\textit{None / Top-1}} &
\multicolumn{2}{c}{\textit{Base Case / Top-1-R}}
\\\hline
\multicolumn{1}{r|}{Type Specification{}} &
\multicolumn{2}{c|}{\textit{Polymorphic}} &
\multicolumn{2}{c}{\textit{Polymorphic}}
\\\hline
\textbf{Name} &
\textbf{Expert} & \textbf{Random} &
\textbf{Expert} & \textbf{Random}
\\
&
& {(50\%, 90\%)} &
& {(50\%, 90\%)}
\\
&&&&\\
&&&&\\
list\_append&3 (75\%)&(3,4)$^{}$&1+1 (100\%)&(1+2,1+4)$^{}$\\
list\_concat&3 (100\%)&(2,3)$^{}$&1+1 (---)&(1+3,1+5)$^{}$\\
list\_drop&4 (80\%)&(6,9)$^{}$&1+2 (100\%)&(1+8,1+19)$^{}$\\
list\_filter&3 (60\%)&\labelBlankHigherOrder&1+2 (60\%)&\labelBlankHigherOrder\\
list\_fold&2 (67\%)&\labelBlankHigherOrder&1+1 (50\%)&\labelBlankHigherOrder\\
list\_last&3 (75\%)&(6,10)$^{}$&1+2 (100\%)&(1+4,1+10)$^{}$\\
list\_length&3 (100\%)&(3,4)$^{}$&1+1 (100\%)&(1+2,1+2)$^{}$\\
list\_map&2 (50\%)&\labelBlankHigherOrder&1+1 (66\%)&\labelBlankHigherOrder\\
list\_pairwise\_swap&\labelColorFailed{\scriptsize{failed}}&\labelColorFailed{\scriptsize{failed}}&\labelColorFailed{\scriptsize{failed}}&\labelColorFailed{\scriptsize{failed}}\\
list\_rev\_append&2 (67\%)&(4,7)$^{}$&1+1 (66\%)&(1+2,1+4)$^{}$\\
list\_rev\_fold&2 (100\%)&(2,4)$^{}$&\labelBlankNonRec&\labelBlankNonRec\\
list\_rev\_snoc&2 (67\%)&(3,8)$^{}$&1+1 (100\%)&(1+3,1+4)$^{}$\\
list\_rev\_tailcall&2 (67\%)&(2,4)$^{}$&1+1 (100\%)&(1+2,1+4)$^{}$\\
list\_snoc&2 (67\%)&(2,4)$^{}$&1+1 (100\%)&(1+2,1+3)$^{}$\\
list\_stutter&2 (100\%)&(2,3)$^{}$&1+1 (100\%)&(1+2,1+2)$^{}$\\
list\_take&3 (60\%)&(6,10)$^{}$&1+3 (100\%)&(1+7,1+15)$^{}$\\
list\_tl&2 (100\%)&(2,3)$^{}$&\labelBlankNonRec&\labelBlankNonRec\\
&&&&\\
&&&&\\
tree\_collect\_leaves&3 (100\%)&(2,3)$^{}$$^{\labelRandomTime{3}}$&1+2 (100\%)&(1+2,1+3)$^{}$\\
tree\_count\_leaves&3 (100\%)&\labelTimeout&1+1 (100\%)&\labelTimeout\\
tree\_count\_nodes&3 (100\%)&(4,6)$^{}$$^{\labelRandomTime{10}}$&1+2 (100\%)&(1+3,1+4)$^{}$$^{\labelRandomTime{3}}$\\
tree\_inorder&3 (75\%)&(3,4)$^{}$&1+2 (100\%)&(1+3,1+3)$^{}$\\
tree\_map&3 (75\%)&\labelBlankHigherOrder&1+2 (75\%)&\labelBlankHigherOrder\\
tree\_preorder&3 (100\%)&(2,4)$^{}$$^{\labelRandomTime{3}}$&1+2 (100\%)&(1+2,1+3)$^{}$\\
&&&&\\

\end{tabular}

\vsepBeforeCaption
  \captionsetup{justification=centering}
  \caption{
    Experiments with Polymorphic Types.
    %% Tasks without polymorphic specifications are marked ``\labelColorSkipped{---}''.
    \\
    \textbf{5a}:
      Percentages w.r.t. to number of examples in Experiment 2a.
    \\
    \textbf{6a}:
      Percentages w.r.t. to total specification size in Experiment 3a.
  }
\label{fig:poly-experiments}
\end{figure}


\end{document}
\endinput
